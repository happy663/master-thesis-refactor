\chapter{おわりに}
\thispagestyle{myheadings}





\section{まとめ}

% % TODO: 構築を行ったでいいのか?
本論文では様々な環境や状況に対応できるPDRベースの3次元屋内位置推定ライブラリの構築を行った.
PDRはスマートフォンなどの機器さえあれば環境に左右されず一定の推定が可能な推定手法である.
一方でPDRは相対的な手法であるため初期位置,初期進行方向が不明な問題や時間の経過に応じて特有の誤差が蓄積する問題がある.
そのため環境情報などを使用して補正するハイブリット手法が用いられる場合が多い.
しかしハイブリット手法は特定の環境を想定したものが多く,複数の環境での適用は難しい.
そこで本研究では様々な環境や状況に対応できるPDRベースの3次元屋内位置推定ライブラリの検討と構築を行った.

具体的にはPDRアルゴリズムの包括的実装を行いモジュール性と拡張性を重視した設計を採用した.
PDRの各要素(ステップ検出,方向推定,軌跡計算)を柔軟に設計しモジュラーな構造を実現した.
TrajectoryCalculatorとPDREstimatorの設計により異なるセンサー設定や環境に適応可能なアルゴリズムを構築した.
特に補正モジュールに関してはDriftCorrectorによる時間経過に伴うドリフト誤差の補正,
MapMatchCorrectorによるフロアマップ情報を用いた
歩行可能座標への補正,WirelessSignalCorrectorによる無線信号を用いた初期進行方向補正など,多様な補正手法を実装した.
これらのモジュールは独立して機能しながらも相互に連携し適用可能である.
また気圧センサーデータを活用した相対的な階層推定により3次元的な位置推定を可能にした.

ライブラリのアルゴリズム構築やその検証のためにxDR Challenge 2023の環境を用いた.
実際の歩行時のトレーニングデータを用いた各処理の流れやそのコード,
推定軌跡の可視化を行い軌跡がどのようにして補正されるかを示した.
評価の結果としてl\_ce,l\_eag,l\_ve,l\_obstacleの項目において90点前後の精度を達成できた.
これは提案手法が基本的なPDR課題に対して効果的に機能しているを示している.
一方で局所空間における円形誤差(l\_ca)は62.51点と著しく低い結果となった.
この結果は実装アルゴリズムが比較的シンプルな構成であり,複雑な環境変化への対応力が限定的であるのを示唆している.
また駅構内と大学キャンパスという異なる環境での適用可能性を検討し,
PDRと環境情報を活用した補正手法の組み合わせにより,様々な環境に対応可能であるのを示した.
本研究で構築したライブラリは各モジュールの独立性を高め,
新たな補正手法や推定アルゴリズムを容易に追加できる設計とし将来的な機能拡張や性能向上の基盤を確立した.


\section{今後の課題}

PDRアルゴリズムの基本性能の向上は引き続き重要な課題である.現状の実装では歩行タイミングの検出に適応的な閾値処理を用いているが,歩行速度や路面状況の急激な変化に対する追従性には改善の余地がある.特に階段や傾斜路での歩行,混雑した環境での不規則な歩行パターンへの対応が不十分である.また方向推定においては地磁気センサとの融合による補正が考えられるが,磁場の歪みが大きい環境での安定性確保が技術的な課題となる.マップマッチングについては現在の決定論的なアプローチに加えて,位置の不確実性を考慮した確率的な手法の導入を検討する必要がある.具体的にパーティクルフィルターやカルマンフィルターなどの状態推定手法の導入により,センサーノイズや環境の不確実性をより適切に扱える可能性がある.

ライブラリ設計面では処理の優先順位付けや条件分岐の制御についての改善が必要である.
現在のビルダーパターンによる設計は補正処理の柔軟な組み合わせを可能としているが,
環境条件の変化に応じた動的な処理の切り替えには対応していない.
またリアルタイムでの位置推定を実現するためには,センサーデータの非同期処理への対応も課題となる.

深層学習技術の導入も検討課題である.xDR Challenge 2023での結果から,
我々の手法では局所的な精度向上に限界があるのが判明した.
ニューラルネットワークを用いた特徴抽出や
環境変化に対する適応学習など,より高度なアプローチの導入が必要となる.
特にCNNやRNNを活用した歩行パターンの認識や軌跡の予測モデルの構築が有望である.
また転移学習やドメイン適応の技術を用いて,
異なる環境間でのモデルの再利用性を高めるのも重要な研究課題となる.
ただし学習データの収集や計算リソースの制約,リアルタイム性の確保など実用化に向けた課題も多い.

評価方法の拡充も重要である.xDR Challenge 2023の環境での評価に加え
商業施設やオフィスビル,地下街など異なる特性を持つ環境での検証が必要となる.
特に長期的な性能評価や多数のユーザーによる同時利用時の性能検証が重要である.
また環境変化に対するロバスト性の定量的評価手法の確立も課題となる.
具体的には温度変化や湿度変化によるセンサー特性の変動,
建物内の什器の配置変更,電波環境の変化など,様々な外乱要因に対する性能評価が必要である.
さらにユーザーの歩行特性やデバイスの保持方法の違いによる影響も系統的に評価する必要がある.

ライブラリの実用性向上も重要な課題である.API設計の改善やドキュメントの充実化を通じて,
ライブラリの利用者がより容易に最適な位置推定システムを構築できる環境を整備する必要がある.
また異なるプラットフォームやセンサーデバイスへの対応,セキュリティやプライバシーの考慮なども重要となる.
さらにオープンソースコミュニティとの協働を通じて,ライブラリの継続的な改善と機能拡張を図る体制の構築も重要である.
これらの課題への総合的に取り組みよってより実用的で信頼性の高い屋内位置推定ライブラリの実現を目指す.

