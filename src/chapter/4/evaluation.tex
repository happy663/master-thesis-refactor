
\section{xDR Challenge 2023 環境での評価}


xDR Challenge 2023の環境についてあらためて詳細に説明する.
対象施設は高速道路のサービスエリアである.
対象の建物は2つあり,1つは2階建て,もう1つは平屋である.
提供される訓練データとスコアリングデータの詳細を表\ref{table:data}に示す.
BLEビーコンとしてMyBeacon(Aplix)\cite{beacon-aplix}を利用し,BLE信号は0.1秒ごとに発信される.
ビーコンの位置は,フロアマップの座標の (x, y, z) 位置として提供される.
競技に使用される9軸IMUセンサデータはAQUOS Sense 6(SHARP)である.
正解のデータはハンドヘルド LiDAR (GeoSLAM ZEB-Horizon) を使用して約 100 Hz で測定されたものである.
サンプリングデータは歩行時間が54秒から667秒までの全323個のデータが提供された.
スコアリングデータは歩行時間が134秒から234秒までの全9個のデータが提供された.
スコアリングはBLEビーコンからのスッキャンデータがないものとあるものでそれぞれ行われる.
9データのうち3個がBLEビーコンのスキャンデータが与えられず,
残りの6個にはBLEビーコンのスキャンデータが与えられる.

\begin{table*}[ht]
    \centering
    \caption{提供データの概要}
    \begin{tabularx}{\textwidth}{|X|l|X|l|l|}
        \hline
        データタイプ & 測定デバイス & レート & 訓練データ & スコアリングデータ \\ \hline
        加速度 & AQUOS Sense 6 & 約100Hz & 使用可能 & 使用可能 \\ \hline
        角速度 & AQUOS Sense 6 & 約100Hz & 使用可能 & 使用可能 \\ \hline
        磁気 & AQUOS Sense 6 & 約100Hz & 使用可能 & 使用可能 \\ \hline
        BLE RSSI & AQUOS Sense 6 & 10Hzのビーコンから送信,AQUOS Sense 6で受信時に記録 & 使用可能 & 使用可能 \\ \hline
        正解位置 (Ground Truth) (x, y, z) & ZEB-Horizon & 約100Hz & 使用可能 & 始めと終わりのみ使用可能 \\ \hline
        正解姿勢 (Ground Truth) (四元数) & ZEB-Horizon & 約100Hz & 使用可能 & 始めと終わりのみ使用可能 \\ \hline
        正解階層名 (Ground Truth) & - & 各パスの1階層名 & 使用可能 & 使用可能 \\ \hline
    \end{tabularx}
    \label{table:data}
\end{table*}

xDR Challenge 2023では,
PDRベンチマーク標準化委員会によって提供された評価フレームワークを使用して評価が行われた.
このフレームワークでは5つの評価指標が用いられ,
それらは円形誤差(l\_ce),局所空間における円形精度(l\_ca),誤差蓄積勾配(l\_eag),速度誤差(l\_ve),
障害物回避要件(l\_obstacle)である.総合評価指標は式\ref{eq:evaluation_index}に示す重み付き和によって計算される.

\begin{equation}
	\begin{aligned}
		I_i = & W_{ce} \times I_{ce} + W_{ca} \times I_{ca}                                        \\
		      & + W_{eag} \times I_{eag} + W_{ve} \times I_{ve} + W_{obstacle} \times I_{obstacle}
	\end{aligned}
	\label{eq:evaluation_index}
\end{equation}

xDR Challengeに出場したスコア上位のチームは深層学習を活用した高度なアプローチを採用しているのに対し,
我々は基本的なPDRにドリフト補正,マップマッチング,
BLEビーコンの多辺測量を組み合わせたより単純で実装が容易なアプローチを採用した.

評価結果では,l\_ce(88.55),l\_eag(93.02),l\_ve(95.55),
l\_obstacle(93.48)において一定の精度を達成できた.
特に速度誤差と障害物回避要件では95点前後の高いスコアを記録し,
基本的なPDRアルゴリズムとフロアマップの補正が効果的に機能しているのが示された.
% また,誤差蓄積勾配の93.02というスコアは,ドリフト補正が適切に機能していることを示している.
しかし,局所空間における円形精度(l\_ca)は62.51点と著しく低い結果となった.
この問題の主な要因として,環境条件の変化やセンサデータの
微細な変動が位置推定結果に過度に影響を与えている可能性がある.
また上位チームと比較して大きな差が生じた理由として,
我々の手法が比較的シンプルな構成であり,
複雑な環境変化への対応力が限定的であるのが挙げられる.

% TODO: 今後の課題に入れてもいいかも
これらの課題を解決するために,以下の改善が必要である.第一に,
センサデータの前処理段階でのノイズ除去とフィルタリングの最適化により,
入力データの品質向上を図る.
第二に,位置推定の不確実性をより効果的に扱うための確率的アプローチの導入を検討する.
具体的には,異なるセンサ情報の特性を統合的に考慮できる推定手法の開発が重要である.
例えば,パーティクルフィルタなどの確率的手法を部分的に導入によって
センサデータの不確実性をより柔軟に扱える.
このアプローチにより,環境変化に対するロバスト性を向上させ,
特に局所空間における位置推定の精度改善が期待できる.ただし,
手法の複雑さと計算コストのバランスを慎重に考慮し,実用性を損なわない範囲で性能向上を目指す必要がある.

今後の研究では,各種センサ情報の特性を深く分析し,
それらを効果的に統合する手法の開発に注力する.
複数のデータソースの長所を活かしつつ,
環境変化に対する頑健性を高め,より高精度で安定した屋内位置推定システムを実現する.


\begin{table}[ht]
	\caption{評価指数の概要}
	\centering
	\begin{tabular}{l|l|l}
		\hline
		指標                        & 値 (\%) & 重み   \\ \hline
		l\_ce(CE:円形誤差)            & 88.55  & 0.25 \\
		l\_ca(CA\_l:局所空間における円形精度) & 62.51  & 0.20 \\
		l\_eag(EAG:誤差蓄積勾配)        & 93.02  & 0.25 \\
		l\_ve(VE:速度誤差)            & 95.55  & 0.15 \\
		l\_obstacle(障害物回避要件)      & 93.48  & 0.15 \\
		l (総合評価指数)                & 86.25  &      \\ \hline
	\end{tabular}
	\label{table:evaluation_index}
\end{table}


