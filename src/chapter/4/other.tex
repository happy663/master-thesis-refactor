
\section{他環境における検討}

本章では,前章で提案した位置推定手法の他環境への適用可能性を検討する.
具体的には,GPSの利用が制限される駅構内と大学キャンパスの2つの異なる環境を対象に,
PDRと環境情報を使用した補正の適用可能性を検討する.


\subsection{駅の構内での検討}
駅構内での位置推定をする場合を考える.
駅は多くの人々が日常的に使用する場所であり,駅を対象として位置推定の研究も行われるなど位置推定の需要が高い.
駅の改札は地上から続いてるものもあれば地下にあるものもある.
地下の場合は特に衛星からの電波が届きにくい場所であるためGPSが有効ではない.
このような環境ではPDRが有効な手法である.
駅の改札の位置は工事などがない限り,基本的に固定位置から変化しない.
改札を通った時の位置をPDRの正解初期座標として使用できる.
改札を通って出た後,乗り換えを行う場合がある.
このような場合は次の改札口を正解補正座標として利用できる.
ICなどを使って駅改札を通った場合,ユーザを一意に識別できる.
そのため特定のユーザが乗り換えをした情報を収集するのは比較的容易である.
正解初期地点と正解補正座標を利用すればListing\ref{lst:drift-corrector}に示したドリフト補正を適用できる.
また全ての駅ではないがある一定規模以上の駅構内の場合フロアマップ情報が入手できる可能性が高い.
その場合フロアマップ情報を用いたListing\ref{lst:rotate-trajectory-using-floormap}のマップマッチング補正が適用できる.

\subsection{大学のキャンパスでの検討}
大学のキャンパスで位置推定をする場合を考える.
大学には屋外環境と屋内環境がある.
建物間の移動経路を把握する場合はGPSが有効である.
しかし大学の建物内での移動経路を把握する場合GPSでは困難である.
このような場合にPDRを軸とした移動経路の把握を検討できる.
大学は研究室やサ―クルなど異なるコミュニティが混在している.
それらは1つの組織が大本で管理しているのではなく個々が独立運営している.
このような場所でBLEビーコンを配置する場合,各コミュニティへの申請のコストや
,場合によっては配置を拒否される可能性がある.
BLEビ―コン以外の電波の利用を考えた場合,Wi-Fiの電波の利用が検討できる.
Wi-Fiの基地局なら基本的にどのコミュニティにも配置がしてあり,設置コストの面でBLEビーコンと比べると低い.
しかし既知のWi-Fiの基地局位置情報の把握はコストが大きい.
そのためこのような場所ではWi-Fiの電波を使ったFP補正が有効だと考えられる.
FPを使った手法なら基地局の位置情報を把握していない場合でも利用できる.
3章ではBLEビ―コンの元にFP処理を行う関数を実装した.
Wi-FiとBLEは通信範囲や消費電力などで異なる点があり,
内部の処理や閾値を変化させる必要はあるが,基本的に与える引数やそのデ―タ形式を揃えれば同様に適用できる.
また部屋に出入りする際には固定の位置の出入り口がある.
個人がそこを出入りした情報を取得できれば,
正解初期座標や正解補正座標としてドリフト除去を適用できる.
