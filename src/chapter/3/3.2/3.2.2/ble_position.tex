
% TODO: 3.段落が短すぎるのでまとめたり増やす必要がある.
\subsubsection{電波強度と送信機の基地局情報を用いた初期進行方向の補正}

% TODO: ここのフォーマットを各章で統一した方がよさそう:要修正
フロアマップ情報を用いた初期進行方向補正は,建物の構造に依存するため,オープンスペースが
多い環境などでは効果的に機能しない場合がある.このような環境でも適用可能な
手法として,本ライブラリでは無線送信機の基地局情報を活用した補正クラス
WirelessSignalCorrectorクラスを提供している.

この手法を利用するために必要な情報は主に2つある.1つ目は歩行者が
移動中に収集した信号スキャンデータである.これは歩行者の
スマートフォンが周辺の無線送信機を検知した際に記録される情報で,
各送信機のID,検知した時刻,そのときの電波強度(RSSI)が含まれている.
歩行者が送信機に近づいたり遠ざかると,この電波強度は
時間とともに変化する.2つ目は送信機の基地局情報である.
これは各送信機のIDとその送信機が実際に設置されている座標が
記録された情報である.図\ref{fig:ble-beacon-position}は,
xDR Challenge 2023環境におけるBLEビーコンの配置例を示している.各送信機は
既知の座標に固定されており,歩行者の移動に伴って受信されるRSSI値が変化する.
この手法の利用例をListing\ref{lst:ble-beacon-position}に示す.
% TODO:手法なのかクラスなのか統一しよう

% TODO: 2.データが1つしかないように見える.複数個あるのを表現した方がいいと思う.
% TODO: 2.captionの名前は検討した方がいいかも
\begin{lstlisting}[caption={WirelessSignalCorrectorの使用例},label=lst:ble-beacon-position,float=h]
# 送信機の基地局情報
transmitter_positions = pd.read_csv('transmitter_positions.csv')
# transmitter_id: "f2:65:d1:87:a4:2c"
# x: 15.2  # メートル
# y: 24.8  # メートル
# floor: "floor_5"

# 歩行中に収集したスキャンデータ
signal_realtime_scans = pd.read_csv('signal_scans.csv')
# ts: 1234567890.123  # タイムスタンプ(秒)
# transmitter_id: "f2:65:d1:87:a4:2c"  # 送信機ID
# rssi: -68  # 電波強度(dBm)

# WirelessSignalCorrectorの初期化と補正の実行
wireless_corrector = WirelessSignalCorrector(
    signal_realtime_scans=signal_realtime_scans,
    transmitter_positions=transmitter_positions,
    rssi_threshold=-70  # 電波強度の閾値(dBm)
)
\end{lstlisting}

\begin{figure}[H]
    \centering
    \includegraphics[width=\linewidth]{../image/ble-beacon-position.jpg}
    \caption{送信機の配置例(xDR Challenge 2023環境)}    \label{fig:ble-beacon-position}
\end{figure}

% TODO: 2.RSSIの値はデータの上位何%という風に決めた方がいいかも.
% 多くのビーコンを使用してなんとなくの全体像を把握するの重要なポイントな気がする

処理の流れとしてまず受信した電波強度に対して閾値処理を行う.
デフォルトではRSSIが-70dBmより強い信号のみを使用する.この値は一般的な無線信号の減衰特性を考慮して
設定されており,およそ3メートル程度の範囲内での受信信号に相当する.
この閾値は環境やユースケースに応じて調整可能であり,WirelessSignalCorrectorの
初期化時にrssi\_thresholdパラメータとして指定できる.例えば
与えられる信号スキャンデータの電波強度が強い割合が小さい場合は-80dBm程度に緩和し,
逆に割合が多い場合は-65dBm程度に厳格化するといった調整が可能である.
この手法の特徴は建物の構造に依存せず,送信機が適切に配置されて
いれば任意の環境で適用可能な点にある.また電波強度の閾値の調整によって
補正の精度と信頼性のバランスを制御できる.ただしこの手法の
効果は送信機の配置密度や,環境内での電波伝搬特性に大きく
影響される点には注意が必要である.

次に選択された各信号について,受信時刻にもっとも近い推定軌跡上の
ポイントを特定する.図\ref{fig:ble-merge}は,この対応付けの結果を
可視化したものである.軌跡上のポイントは時間経過に応じて色付けされており,
青色のポイントは対応する送信機の位置を示している.

最後にこれらの対応点間の距離の総和が最小となるように軌跡全体の回転角度を最適化する.
受信時刻$t$における軌跡上の点と対応するビーコン基地局との距離の総和$D(\theta)$を式\ref{eq:distance_sum}で表す.
ここで$(x_t(\theta), y_t(\theta))$は角度$\theta$で回転させた軌跡上の時刻$t$における座標,$(b_x^t, b_y^t)$は対応するBLEビーコン基地局の座標を表す.
式\ref{eq:opt}に示すように,最適な回転角度$\theta_{\mathrm{opt}}$は距離の総和$D(\theta)$を
最小化する角度として定義される.
この最適化問題はグリッドサーチにより$[0, 2\pi]$の範囲で最適解を探索する.%TODO: この表現微妙かも
この探索によりBLEビーコン基地局の位置と推定軌跡の位置関係がもっとも
整合する角度を見つけ最適な初期進行方向を決定できる.

\begin{equation}
\label{eq:distance_sum}
D(\theta) = \sum_{t} \sqrt{(x_t(\theta) - b_x^t)^2 + (y_t(\theta) - b_y^t)^2}
\end{equation}
\begin{equation}
\label{eq:opt}
\theta_{\mathrm{opt}} = \arg\min_{\theta \in [0, 2\pi]} D(\theta)
\end{equation}

\begin{figure}[H]
	\centering
	\includegraphics[width=\linewidth]{../image/ble-merge.jpg}
	\caption{BLEビーコンの基地局の基地局情報}    \label{fig:ble-merge}
\end{figure}







