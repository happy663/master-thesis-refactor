
\subsubsection{電波フィンガープリントを用いた初期進行方向の補正}
% TODO:2. FPなのかフィンガープリントなのか統一した方がいいかも

基地局情報を用いた補正手法は効果的である一方,基地局の
正確な位置情報を得るのが困難な場合がある.このような状況に対応するため,
本ライブラリでは電波強度フィンガープリント(FP)を用いた補正手法も提供している.
この手法は,WirelessSignalCorrectorクラスのFPメソッドとして実装されている.

この手法の利用例をListing\ref{lst:rotate-trajectory-using-ble-fingerprint}に示す.
この手法を利用するために必要な情報は2つある.1つ目は前節と同様の
歩行者が移動中に収集した信号スキャンデータである.2つ目は
フィンガープリントデータである.これは事前に環境内の様々な位置で
計測された各送信機のID,電波強度,およびその計測位置の座標が
記録されたデータベースである.前節とは異なり,この手法では基地局の
正確な位置情報は必要としない.

% TODO:3 WiressSignalCorrectorが被ってるのでキャプション名を変更した方がいいかも
\begin{lstlisting}[caption={WirelessSignalCorrectorの使用例},label=lst:rotate-trajectory-using-fingerprint,float=ht]
# フィンガープリントデータ
signal_fingerprints = pd.read_csv('fingerprints.csv')
# ts: 1234567890.123  # 計測時刻(秒)
# x: 15.2            # 計測位置のx座標(メートル)
# y: 24.8            # 計測位置のy座標(メートル)
# transmitter_id: "f2:65:d1:87:a4:2c"  # 送信機ID
# rssi: -68          # 電波強度(dBm)
# floor: "floor_5"   # フロア名

# WirelessSignalCorrectorの初期化と補正の実行
corrector = WirelessSignalCorrector(
    signal_realtime_scans=signal_realtime_scans,
    signal_fingerprints=signal_fingerprints,
    signal_threshold=-70
)
\end{lstlisting}


WirelessSignalCorrectorのFPを用いた補正処理は,歩行者から受信した各無線信号について,
受信時刻$t$ごとに位置推定を行う.
各時刻において,受信した送信機IDと同じデータポイントをFPデータベースから抽出する.
各データポイントの重み$w$は,受信時刻$t$におけるRSSI値$r_t$と,
FPデータベース内の同じビーコンのRSSI値$r_f$との差に基づいて,式\eqref{eq:weight}で計算される.
ここで,$\sigma$は電波強度の標準偏差を表すパラメータであり,
環境に応じて調整可能である.この式はガウシアンカーネルに基づいており,
RSSI値の差が小さいほど大きな重みが与えられる.

次に,これらの重みを用いて時刻$t$における推定位置$(p_x^t, p_y^t)$を式\eqref{eq:position}で計算する.
ここで,$(x_i, y_i)$はFPデータベース内の各データポイントの座標を表し,$
N_t$は時刻$t$において抽出されたデータポイントの総数である.
この重み付き平均により,その時刻のRSSI値と類似したデータポイントの座標がより強く反映された推定位置が得られる.
最適な回転角度の決定には,3.2.4節で導入した距離の総和$D(\theta)$を用いる.
ここでは,固定点として式\eqref{eq:position}で推定された位置$(p_x^t, p_y^t)$を使用する.
最適な回転角度$\theta_{\mathrm{opt}}$は,前節と同様に距離の総和を最小化する角度として決定される.
この手法の特徴は,基地局の正確な位置情報を必要としない点である.
その代わりに,事前に環境内で十分なFPデータを収集する必要がある.
ただし,環境の物理的な変化や人の往来,電波干渉などの要因により,FPデータは時間とともに劣化する可能性がある.
そのため,定期的なFPデータの更新が必要である.

\begin{equation}
\label{eq:weight}
w_t = \exp\left(-\frac{(r_t - r_f)^2}{2\sigma^2}\right)
\end{equation}

\begin{equation}
\label{eq:position}
p_x^t = \frac{\sum_{i=1}^{N_t} w_{t,i} x_i}{\sum_{i=1}^{N_t} w_{t,i}}, \quad
p_y^t = \frac{\sum_{i=1}^{N_t} w_{t,i} y_i}{\sum_{i=1}^{N_t} w_{t,i}}
\end{equation}

% TODO:2.パスロスモデルの比重も増やせる説明をした方がいいかも
% TODO:2.5 位置を推定した座標の図が欲しい
% TODO:2. 全体像を捉えて方向補正をするのが目的.なのでなんとなくこの辺にいそうというのが重要.
% 正確な位置を推定する必要はないといった方がいいかも
% TODO:2. 軌跡全体をみて最適化する形式だから十分なFPデータがなくても成り立つみたいな文が欲しい

% \begin{figure}[H]
%     \centering
%     \includegraphics[width=\linewidth]{../image/fingerprint-rotate.jpg}
%     \caption{BLEのFPを用いた補正後の軌跡}    \label{fig:fingerprint-rotate}
% \end{figure}










